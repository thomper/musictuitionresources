\documentclass{article}
\usepackage{fontspec}
\usepackage{lilyglyphs}
\usepackage{gchords}
\usepackage{array}
\begin{document}
\setlength{\parindent}{0pt}


\section{Adult Beginner Chord Chart (Naturals Only)}

Remember:
\begin{itemize}
    \item If no chord type is specified then it's a major chord, e.g. C = Cmaj
    \item A lower case m after the chord means minor
    \item \textit{X}5 is also called a power chord
\end{itemize}

\def\numfrets{5}
\mediumchords

\begin{table}[htp]
\centering
\begin{tabular}{|l|l|l|}
\hline
\chords{\chord{t}{x,f3p3,f2p2,o,f1p1,o}{C}} \chords{\chord{t}{x,x,x,f3p5,f2p4,f1p3}{Cm}} & &
    \chords{\chord{t}{x,o,f1p2,f2p2,f3p2,o}{A}} \chords{\chord{t}{x,o,f2p2,f3p2,f1p1,o}{Am}} \\ \hline

\chords{\chord{t}{x,x,o,f2p2,f3p3,f1p2}{D}} \chords{\chord{t}{x,x,o,f2p2,f3p3,f1p1}{Dm}} & &
    \chords{\chord{t}{x,x,f1p4,f2p4,f3p4,x}{B}} \chords{\chord{t}{x,x,x,f3p4,f2p3,f1p2}{Bm}} \\ \hline

\chords{\chord{t}{o,f3p2,f2p2,f1p1,o,o}{E}} \chords{\chord{t}{o,f3p2,f2p2,o,o,o}{Em}} & & \\ \hline

\chords{\chord{t}{x,x,x,f2p2,f1p1,f1p1}{F}} \chords{\chord{t}{x,x,x,f1p1,f1p1,f1p1}{Fm}} & &
    \chords{\chord{t}{o,p2,x,x,x,x}{E5}} \chords{\chord{t}{x,o,p2,x,x,x}{A5}} \\ \hline

\chords{\chord{t}{f2p3,f1p2,o,o,f3p3,f4p3}{G}} \chords{\chord{t}{x,x,x,f1p3,f1p3,f1p3}{Gm}} & &
    \chords{\chord{t}{f1p1,f5p3,x,x,x,x}{F5}} \chords{\chord{t}{x,f1p2,f5p4,x,x,x}{B5}} \\ \hline

\end{tabular}
\end{table}

The power chords without any open strings ringing (\textbf{F5} and \textbf{B5} here) can be moved up and down the neck to play any power chord!  Power chords are neither major nor minor.  They're used a lot in punk.


\end{document}
