\documentclass{article}
\usepackage{fontspec}
\usepackage{lilyglyphs}
\usepackage{gchords}
\usepackage{array}
\begin{document}
\setlength{\parindent}{0pt}


\section{Major Scale Chord Chart}

Each row is a major scale.  The row headings show the root (aka tonic) note for the scale.  The column headings show the scale degree for each chord in the scale.

\bigskip

Remember:
\begin{itemize}
    \item If no chord type is specified then it's a major chord, e.g. C = Cmaj
    \item A lower case m after the chord means minor
    \item A \textsuperscript{\circ} after the chord means diminshed
\end{itemize}

\renewcommand{\arraystretch}{2.2}
\begin{table}[htp]
\centering
\begin{tabular}{|l|l|l|l|l|l|l|l|}
\hline
& \textbf{I} & \textbf{ii} & \textbf{iii} & \textbf{IV} & \textbf{V} & \textbf{vi} & \textbf{vii\textsuperscript{\circ}} \\ \hline
\textbf{C} & C & Dm & Em & F & G & Am & B\textsuperscript{\circ}  \\ \hline
\textbf{C\textsuperscript{\sharp{}}} & C\textsuperscript{\sharp{}} & D\textsuperscript{\sharp{}}m & Fm & F\textsuperscript{\sharp{}} & G\textsuperscript{\sharp{}} & A\textsuperscript{\sharp{}}m & C\textsuperscript{\circ}  \\ \hline
\textbf{D} & D & Em & F\textsuperscript{\sharp{}}m & G & A & Bm & C\textsuperscript{\sharp{}}\textsuperscript{\circ}  \\ \hline
\textbf{D\textsuperscript{\sharp{}}} & D\textsuperscript{\sharp{}} & Fm & Gm & G\textsuperscript{\sharp{}} & A\textsuperscript{\sharp{}} & Cm & D\textsuperscript{\circ}  \\ \hline
\textbf{E} & E & F\textsuperscript{\sharp{}}m & G\textsuperscript{\sharp{}}m & A & B & C\textsuperscript{\sharp{}}m & D\textsuperscript{\sharp{}}\textsuperscript{\circ}  \\ \hline
\textbf{F} & F & Gm & Am & A\textsuperscript{\sharp{}} & C & Dm & E\textsuperscript{\circ}  \\ \hline
\textbf{F\textsuperscript{\sharp{}}} & F\textsuperscript{\sharp{}} & G\textsuperscript{\sharp{}}m & A\textsuperscript{\sharp{}}m & B & C\textsuperscript{\sharp{}} & D\textsuperscript{\sharp{}}m & F\textsuperscript{\circ}  \\ \hline
\textbf{G} & G & Am & Bm & C & D & Em & F\textsuperscript{\sharp{}}\textsuperscript{\circ}  \\ \hline
\textbf{G\textsuperscript{\sharp{}}} & G\textsuperscript{\sharp{}} & A\textsuperscript{\sharp{}}m & Cm & C\textsuperscript{\sharp{}} & D\textsuperscript{\sharp{}} & Fm & G\textsuperscript{\circ}  \\ \hline
\textbf{A} & A & Bm & C\textsuperscript{\sharp{}}m & D & E & F\textsuperscript{\sharp{}}m & G\textsuperscript{\sharp{}}\textsuperscript{\circ}  \\ \hline
\textbf{A\textsuperscript{\sharp{}}} & A\textsuperscript{\sharp{}} & Cm & Dm & D\textsuperscript{\sharp{}} & F & Gm & A\textsuperscript{\circ}  \\ \hline
\textbf{B} & B & C\textsuperscript{\sharp{}}m & D\textsuperscript{\sharp{}}m & E & F\textsuperscript{\sharp{}} & G\textsuperscript{\sharp{}}m & A\textsuperscript{\sharp{}}\textsuperscript{\circ}  \\ \hline
\end{tabular}
\end{table}

\clearpage

\section{Exercise: Place Chords Within A Scale}

\subsection{Wonderwall (Simplified)}


Artist: Oasis

Album: (What's the Story) Morning Glory?

Year: 1995

Genre: Britpop

\textbf{Key: G}

\def\numfrets{5}
\chords{
    \chord{t}{n,p2,p2,n,n,n}{Em \rule{0.5cm}{0.15mm}}
    \chord{t}{p3,p2,n,n,p3,p3}{G \rule{0.5cm}{0.15mm}}
    \chord{t}{x,x,n,p2,p3,p2}{D \rule{0.5cm}{0.15mm}}
    \chord{t}{x,n,p2,p2,p3,n}{Asus4 \rule{0.3cm}{0.15mm}}
    \chord{t}{x,p3,p2,n,p1,n}{C \rule{0.5cm}{0.15mm}}
}
\bigskip

Use the row for the \textbf{G major scale} in the major scale chart to match chords to their roman numeral scale degrees.

\bigskip

Write the answers in the spaces provided to the right of each chord name.

\bigskip

Because we haven't met the suspended chord yet, treat Asus4 as though it was an A minor chord.


\subsection{Have You Ever Seen The Rain (Simplified)}

Artist: Creedence Clearwater Revival

Album: Pendulum

Year: 1971

Genre: Country Rock

\textbf{Key: C}

\def\numfrets{5}
\chords{
    \chord{t}{x,n,p2,p2,p1,n}{Am \rule{0.5cm}{0.15mm}}
    \chord{t}{x,p3,p2,n,p1,n}{C \rule{0.5cm}{0.15mm}}
    \chord{t}{x,p3,p3,p2,p1,p1}{F \rule{0.5cm}{0.15mm}}
    \chord{t}{p3,p2,n,n,p3,p3}{G \rule{0.5cm}{0.15mm}}
}
\bigskip

Use the row for the \textbf{C major scale} in the major scale chart to match chords to their roman numeral scale degrees.

\clearpage

\subsection{Kindergarten}

Artist: Chloe Moriondo

Album: N/A (single release only)

Year: 2019

Genre: Bedroom Pop

\textbf{Key: G}

\def\numfrets{5}
\chords{
    \chord{t}{p3,p2,n,n,p3,p3}{G \rule{0.5cm}{0.15mm}}
    \chord{t}{n,p2,p2,n,n,n}{Em \rule{0.5cm}{0.15mm}}
    \chord{t}{x,p3,p2,n,p1,n}{C \rule{0.5cm}{0.15mm}}
    \chord{t}{x,p3,p1,x,x,x}{Cm ???}
}
\bigskip

Use the row for the \textbf{G major scale} in the major scale chart to match notes to their roman numeral scale degrees.

\bigskip

Why does this song have both a C major and a C minor chord?  Sometimes artists use chords from outside the major scale to create a particular effect.  Playing a major chord followed immediately by its minor equivalent is a common device.  Other examples of this include `Creep' by Radiohead and `Idiot Wind' by Bob Dylan.

\subsection{Your Song}

Artist: \rule{5cm}{0.15mm}

Year: 2020

\textbf{Key: \rule{0.5cm}{0.15mm}}

\def\numfrets{5}
\chords{
    \chord{t}{n,n,n,n,n,n}{\rule{0.5cm}{0.15mm} \rule{0.5cm}{0.15mm}}
    \chord{t}{n,n,n,n,n,n}{\rule{0.5cm}{0.15mm} \rule{0.5cm}{0.15mm}}
    \chord{t}{n,n,n,n,n,n}{\rule{0.5cm}{0.15mm} \rule{0.5cm}{0.15mm}}
    \chord{t}{n,n,n,n,n,n}{\rule{0.5cm}{0.15mm} \rule{0.5cm}{0.15mm}}
}
\bigskip

Now it's your turn:

\begin{enumerate}
    \item Pick a key (a row) from the major scale chord chart and write it in the space above next to \textbf{Key:}
    \item Make up a song using only chords in that key (no more than four)
    \item Fill in the chord diagrams and chord names above
    \item Fill in the scale degrees next to the chord names
\end{enumerate}

\end{document}
